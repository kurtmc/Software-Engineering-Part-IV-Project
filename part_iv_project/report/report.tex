% Form slides:
% no more than 5 pages
% two columns
% title and originality statement are not counted
\documentclass[twocolumn]{article}
\usepackage{datetime}
\usepackage{hyperref}

\begin{document}

% Title
\pagenumbering{gobble}
\begin{titlepage}


\vspace*{15em}


\centering

{\LARGE
Department of Electrical \& Computer Engineering \\
Final Year Research Project 2015, Interim Report}

\hspace{2em}

% notes on latex tables use "&" as colum sperator
\begin{table*}[h]
\centering
\begin{tabular}{ll}
Project Title: & Preventing plagiarism during practical tests \\
Project Number: & 11 \\
Supervisor Name: & Dr Nasser Giacaman \\
Second Examiner Name: & Avinash Malik \\
Your Name: & Kurt McAlpine \\
Your UID: & 2004750 \\
Partner Name: & Conor Simmonds \\
Date submitted: & 11\textsuperscript{th} May 2015 \\

\end{tabular}
\end{table*}
\begin{table}


\end{table}
\pagebreak

\vspace*{25em}

{\Large Declaration of Originality}

\hspace{5em}

This report is my own unaided work and was not copied from 
nor written in collaboration with any other person.

Name: Kurt McAlpine


\end{titlepage}



\pagenumbering{arabic}

\begin{abstract}

\end{abstract}

% Introduction:
% - project scope
% - to what extent
% * go from general to specific
% * software/software engineering is important
% * programming is difficult
% * struggling in class can cause plaigirism, dropping out
% - contributions this projects makes to the world
% - overview of report
\section{Introduction}
In the world today programming and programs are prevalent and abundant in our
every day lives. So it follows that programming is an important skill. It's
widely accepted that programming is a difficult skill to learn
\cite{jenkins2002difficulty, robins2003learning}. There is a need to innovate
and come up with some solutions to help students be more effective in class.
When students struggle in class with tests and assignments they can be lead to
plagiarism or even dropping out\cite{bennedsen2007failure}. Currently in many
institutions, software engineering courses test their students on their
knowledge and skill in a way that is artificial to real world programming,
namely tests are done with pen and paper and without access to computers with
IDE`s, compilers or the internet. So in this project we want to come up with a
way to give software engineering students robust testing environments that allow
students to have a more realistic set of tools during the test. However whilst
giving students these tools we would also like to give the teaching staff
confidence that plagiarism can be detected automatically without tedious manual
inspection of student submissions as well as detecting plagiarised work that the
student has attempted to conceal.
% TODO overview of report

% Background: (contains technical aspects)
% * what is eclipse
% * mention some of the plaigirism tools that exist, mention how the only analyse end result
% - ACP (some plugin) exists but it lacks features
\section{Background}
Eclipse is an IDE (integrated development environment) that is written is Java.
It can be used to develop in a variety of programming languages. And it also has
a plugin architecture which allows other developers to extend it's functionality
without changing it's core implementation.

There are several plagiarism detection tools available for institutions to use,
two of which are MOSS\cite{schleimer2003winnowing} and
JPlag\cite{lutz2000jplag}. Both these systems are used on submissions from
students to help identify any similarities between students. If the student
submissions are too similar these tools could detect that and warn the lecturer
that plagiarism is the likely cause of this similarity.

A tool called ACP (Active Classroom Programmer)\cite{giacaman2015active} is used
at the University of Auckland in some SOFTENG 701 and 751. ACP is an eclipse
plugin that allows the lecturer to create eclipse workspaces and upload
snapshots of the current state to a webserver. For example during a lecture a
task can be completed in parts and once each part is complete a snapshot of the
code can be uploaded. The students, during the lecture can download these
snapshots using the ACP eclipse plugin, onto thier laptops and make
modifications and experiment with the workspace, they can also attempt to
complete the next part before the lecturer has completed it. Often there are
short breaks during the lecture that allow students to attempt the tasks.

Active Test Programmer (ATP) is an extention to ACP which allows the lecturer to
setup a programming test. The programming test can be distributed to students
before the test starts. The test is encrypted so a secret code is needed to
decrypt the tests. When a secret code is distributed to the students they may
type it in and begin the test. After a predetermined amount of time the the test
is over and the students must submit their code through the eclipse plugin.

% Active Test Programmer:
% - no implementation details
% - features
% - plaigirism module
% - teacher visualisation tool
\section{Active Test Programmer}
For my project, we are going to extend ATP to give lecturers confidence that
plaigirism that has occured during the test can be detected automatically. The
way this will be done is the history of the student completing the test will be
recorded and when the test is submitted the history will go with it. This
history gives us much richer insight into the characteristics of the students
work. Analysis of the history of the students submission will allow for more
robust detection of potential plaigirism.

\subsection{Features} \label{sec:Features}
Create tests: The lecturer will be able to create a programming test in their
IDE and upload it. Once the test is uploaded the lecturer will be given a secret
code that can be used to begin the test.

Download tests: Students will be able to download tests from within their IDE,
they may not begin the test until they have entered the secret code to decrypt
the test. This allows all the students to download and be ready for the test, so
that everyone may begin at the same time, and no students will feel
disadvantaged because it took them longer to download the test, or if wifi is
unavailable the test may be distributed on a USB flash drive.

Upload completed test: Students may upload their completed test through their
IDE. If WiFi is unavailable they may export their completed test to an archived
file and copy it to a flash drive to give to the lecturer. The archived
completed test will contain information about when it was exported so that we
can be sure they completed the test in the given time.

Implementation history: The completed test that the student uploads, will
contain a complete history in snapshots over time of how the test was
implemented. This gives a rich insight into potential plaigirism in the history
of the code.

Viewing reports: Reports can be generated that contain analyses of student
tests. In the analyses it will contain a score of how likely it was that the
students submission contains plaigirised work. The history also gives us the
opportunity to go beyond just checking for plaigirism. In fact a more general
analysis could give us information about where the student may have been
struggling during the test by looking at how long it took a student took to
implement a particular method or class.

% Interim Report:
% - technologies considered, pros/cons (why eclipse for instance)
% - related work (research)
\section{Technologies Considered and Related Work}
% libgit2, eclipse because ACP already developed, research???
For this project the client appliction will be implemented using the Eclipse
plugin framework. There are other IDE's that could potentially be used because
it also have a plugin architecture but Eclips was choosen because ACP is already
implemented for Eclipse and it's going to be extended to contain the features
outlined in \nameref{sec:Features}.


% Implementation:
\section{Implementation}
% libgit2, eclipse text listener, prototype

% Evaluation:
% - students
% - how accurate
% - performance metrics
\section{Evaluation}

% Conclusions
\section{Conclusions}
% Some conclusion. Solution can be used in a more general sense, not just
% related to plaigirism

% References
\bibliographystyle{IEEEtran}
\bibliography{references}

\end{document}

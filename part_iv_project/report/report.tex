% Form slides:
% no more than 5 pages
% two columns
% title and originality statement are not counted
\documentclass[twocolumn]{article}
\usepackage{datetime}

\begin{document}

% Title
\pagenumbering{gobble}
\begin{titlepage}


\vspace*{15em}


\centering

{\LARGE
Department of Electrical \& Computer Engineering \\
Final Year Research Project 2015, Interim Report}

\hspace{2em}

% notes on latex tables use "&" as colum sperator
\begin{table*}[h]
\centering
\begin{tabular}{ll}
Project Title: & Preventing plagiarism during practical tests \\
Project Number: & 11 \\
Supervisor Name: & Dr Nasser Giacaman \\
Second Examiner Name: & Avinash Malik \\
Your Name: & Kurt McAlpine \\
Your UID: & 2004750 \\
Partner Name: & Conor Simmonds \\
Date submitted: & 11\textsuperscript{th} May 2015 \\

\end{tabular}
\end{table*}
\begin{table}


\end{table}
\pagebreak

\vspace*{25em}

{\Large Declaration of Originality}

\hspace{5em}

This report is my own unaided work and was not copied from 
nor written in collaboration with any other person.

Name: Kurt McAlpine


\end{titlepage}



\pagenumbering{arabic}

\begin{abstract}

\end{abstract}

% Introduction:
% - project scope
% - to what extent
% * go from general to specific
% * software/software engineering is important
% * programming is difficult
% * struggling in class can cause plaigirism, dropping out
% - contributions this projects makes to the world
% - overview of report
\section{Introduction}
In the world today programming and programs are prevalent and abundant in our
every day lives. So it follows that programming is an important skill. It's
widely accepted that programming is a difficult skill to learn
\cite{jenkins2002difficulty, robins2003learning}. There is a need to innovate
and come up with some solutions to help students be more effective in class.
When students struggle in class with tests and assignments they can be lead to
plagiarism or even dropping out\cite{bennedsen2007failure}. Currently in many
institutions, software engineering courses test their students on their
knowledge and skill in a way that is artificial to real world programming,
namely tests are done with pen and paper and without access to computers with
IDE`s, compilers or the internet. So in this project we want to come up with a
way to give software engineering students robust testing environments that allow
students to have a more realistic set of tools during the test. However whilst
giving students these tools we would also like to give the teaching staff
confidence that plagiarism can be detected automatically without tedious manual
inspection of student submissions as well as detecting plagiarised work that the
student has attempted to conceal.

% Background: (contains technical aspects)
% - what is eclipse
% - mention some of the plaigirism tools that exist, mention how the only analyse end result
% - ACP (some plugin) exists but it lacks features
\section{Background}
Eclipse is an IDE (intergrated development environment) that is written is Java.
It can be used to develop in a variety of programming languages. And it also has
a plugin architecture which allows other developers to extend it's functionality
without changing it's core implementation.

% Active Test Programmer:
% - no implementation details
% - features
% - plaigirism module
% - teacher visualisation tool
\section{Active Test Programmer}

% Implementation:
\section{Implementation}

% Interim Report:
% - technologies considered, pros/cons (why eclipse for instance)
% - related work (research)
\section{Technologies Considered and Related Work}

% Evaluation:
% - students
% - how accurate
% - performance metrics
\section{Evaluation}

% Conclusions
\section{Conclusions}
Some conclusion.

% References
\bibliographystyle{IEEEtran}
\bibliography{references}

\end{document}
